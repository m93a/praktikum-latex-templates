\documentclass[10pt,a4paper]{article}
\usepackage[utf8]{inputenc}
\usepackage[czech]{babel}
\usepackage[T1]{fontenc}
\usepackage{amsmath}
\usepackage{amsfonts}
\usepackage{amssymb}
\usepackage{amsthm}
\usepackage{graphicx}\usepackage{lmodern}
\usepackage[top = 2cm, bottom = 2cm, left = 2cm, right = 2cm]{geometry}
\usepackage{lettrine}
\usepackage{titlesec}
\usepackage{pdfpages}
\usepackage{url}
\usepackage{hyperref}
\usepackage{booktabs}
\usepackage{caption}
\usepackage{microtype}

%For title page
\usepackage{setspace} %Rámeček nahoře
\usepackage{framed} %Rámeček nahoře
\usepackage{array} %Tabulka dole

%\usepackage{jindra_basic}
\usepackage{titling}
\setlength{\droptitle}{-4\baselineskip}
\pretitle{\begin{center}\huge\bfseries}
\posttitle{\end{center}}
\renewcommand\thesection{\Roman{section}}
\renewcommand\thesubsection{\arabic{subsection}}
\titleformat{\section}[block]{\large\scshape\centering}{\thesection.}{1em}{}
\titleformat{\subsection}[block]{\bfseries \large}{\thesubsection.}{1em}{}

\usepackage{float}
\newfloat{Graf}{htbp}{aux}

\graphicspath{{obr}}

\begin{document}


\thispagestyle{empty}
\newgeometry{top = 2.5cm, bottom = 0cm, left = 2.5cm, right = 3cm}

{ % V tomhle je uzavřena celá titulka
%Tloušťka rámečku
\setlength{\fboxrule}{1.5pt}

\noindent
\framebox{
\begin{minipage}{\textwidth}
\setlength{\parindent}{17.62482 pt}
\phantom{d}

\begin{minipage}{0.6\textwidth}
{
\Large Kabinet výuky obecné fyziky, UK MFF\\
}
\vspace*{0.2cm}

{
\bfseries
\huge Fyzikální praktikum III
}
\end{minipage}
\begin{minipage}{0.4\textwidth}
\begin{center}
\includegraphics[width=4.5cm]{ZFP.jpg}
\end{center}
\end{minipage}\\\\

%\vspace*{0.5cm}

{
\setstretch{1.5}
\Large
\noindent
Úloha č. <++>

\noindent
Název úlohy: <++>\\
\noindent
Jméno: Jindřich Dušek
\hspace*{\fill}
Obor: FOF

\noindent
Datum měření: <++>
\hspace*{\fill}
Datum odevzdání: <++>

\phantom{d}
}
\end{minipage}
}
%Konec horního rámečku

{
\phantom{d}

\Large
Připomínky opravujícího:\\
\vspace*{6.75cm}
}

{
\newcommand{\linka}{\noalign{\hrule height 1pt}}
\newcommand{\linkadva}{\noalign{\hrule height 1.5pt}}
\setlength\extrarowheight{9.5pt}
\Large
\noindent
\begin{tabular}{!{\vrule width 1.5pt} l !{\vrule width 1pt} c !{\vrule width 1pt} c !{\vrule width 1.5pt}}
\linkadva
   & Možný počet bodů & Udělený počet bodů \\\linkadva
  Teoretická část & 0-2 &  \\\linka
  Výsledky a zpracování měření & 0-9 &  \\\linka
  Diskuse výsledků & 0-4 &  \\\linka
  Závěr & 0-1 &  \\\linka
  Použitá literatura & 0-1 &  \\\linkadva
  \hspace*{\fill} \textbf{Celkem} \hspace*{\fill}& max. 17 &  \\
\linkadva
\end{tabular}
}
\phantom{d}

Posuzoval: \hspace*{\fill}dne:~~~~~~~~~~~~~~~~~

}
\newpage
\newgeometry{top = 2cm, bottom = 2cm, left = 2cm, right = 2cm}
\setcounter{page}{1}


%-------------------- ZAČÁTEK PROTOKOLU ------------------------

\section{Úkol}
\begin{enumerate}
\item <++>

\end{enumerate}

\section{Teorie}


\section{Výsledky a zpracování měření}


\section{Diskuse}



\section{Závěr}





\section{Literatura}
\begin{itemize}
\item[[1\hspace*{-0.18cm}]] Kolektiv ZFP KVOF MFF UK, \emph{15. Charakteristiky triody}, \url{https://physics.mff.cuni.cz/vyuka/zfp/_media/zadani/texty/txt_215.pdf}, 11.\,11. 2019
\item[[2\hspace*{-0.18cm}]] B. VYBÍRAL, \emph{Zpracování dat fyzikálních měření}, Hradec Králové: MAFY, 2002
%\item[[4\hspace*{-0.18cm}]] Wikimedia Foundation, \emph{Avogadrova konstanta}, \url{https://cs.wikipedia.org/wiki/Avogadrova_konstanta}, 9.\,4. 2019
\end{itemize}





\end{document}

